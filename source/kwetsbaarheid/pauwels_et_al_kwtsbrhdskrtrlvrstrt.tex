\PassOptionsToPackage{english,french,dutch}{babel}
\documentclass[twoside]{extreport}
\usepackage{inbo_report}
\usepackage[colorspec=0.9, fontsize=0.1\paperwidth, angle=90, hpos=5mm, anchor=lm]{draftwatermark}
\DraftwatermarkOptions{text={ONTWERP}}
\reviewer{
\href{https://orcid.org/0000-0003-2732-7428}{Johan Coeck \includegraphics[height=\fontsizebase]{orcid.eps}}
}
\corresponding{ine.pauwels@inbo.be}
\doi{!!! missing DOI !!!}
\shortauthor{Pauwels, I.; Van Wichelen, J. \& Wils, C.}
\citetitle{Kwetsbaarheidskaart riooloverstorten}

\establishment{Brussel}

\flandersfont{}
\usepackage{reportfont}

\makeatletter
\AtBeginDocument{%
   \expandafter\renewcommand\expandafter\section\expandafter
     {\expandafter\@fb@secFB\section}%
   \newcommand\@fb@secFB{\FloatBarrier
   \gdef\@fb@afterHHook{\@fb@topbarrier \gdef\@fb@afterHHook{}}}
   \g@addto@macro\@afterheading{\@fb@afterHHook}
   \gdef\@fb@afterHHook{}
}
\makeatother

\makeatletter
\AtBeginDocument{%
   \expandafter\renewcommand\expandafter\subsection\expandafter
     {\expandafter\@fb@subsecFB\subsection}%
   \newcommand\@fb@subsecFB{\FloatBarrier
   \gdef\@fb@afterHHook{\@fb@topbarrier \gdef\@fb@afterHHook{}}}
   \g@addto@macro\@afterheading{\@fb@afterHHook}
   \gdef\@fb@afterHHook{}
}
\makeatother

\makeatletter
\AtBeginDocument{%
   \expandafter\renewcommand\expandafter\subsubsection\expandafter
     {\expandafter\@fb@subsubsecFB\subsubsection}%
   \newcommand\@fb@subsubsecFB{\FloatBarrier
   \gdef\@fb@afterHHook{\@fb@topbarrier \gdef\@fb@afterHHook{}}}
   \g@addto@macro\@afterheading{\@fb@afterHHook}
   \gdef\@fb@afterHHook{}
}
\makeatother

\title{Kwetsbaarheidskaart riooloverstorten}

\author{
Ine Pauwels, Jeroen Van Wichelen, Carine Wils
}
\colofonauthor{
\href{https://orcid.org/0000-0002-2856-8787}{Ine Pauwels \includegraphics[height=\fontsizebase]{orcid.eps}}, \href{https://orcid.org/0000-0002-0255-9591}{Jeroen Van
Wichelen \includegraphics[height=\fontsizebase]{orcid.eps}}, \href{https://orcid.org/0000-0002-5626-6786}{Carine Wils \includegraphics[height=\fontsizebase]{orcid.eps}}
}



\colophon{1}
\public{1}
\pagefootmessage{!!! missing DOI !!!}


% Alter some LaTeX defaults for better treatment of figures:
% See p.105 of "TeX Unbound" for suggested values.
% See pp. 199-200 of Lamport's "LaTeX" book for details.
%   General parameters, for ALL pages:
\renewcommand{\topfraction}{0.9}	% max fraction of floats at top
\renewcommand{\bottomfraction}{0.8}	% max fraction of floats at bottom
%   Parameters for TEXT pages (not float pages):
\setcounter{topnumber}{2}
\setcounter{bottomnumber}{2}
\setcounter{totalnumber}{4}     % 2 may work better
\setcounter{dbltopnumber}{2}    % for 2-column pages
\renewcommand{\dbltopfraction}{0.9}	% fit big float above 2-col. text
\renewcommand{\textfraction}{0.07}	% allow minimal text w. figs
%   Parameters for FLOAT pages (not text pages):
\renewcommand{\floatpagefraction}{0.7}	% require fuller float pages
% N.B.: floatpagefraction MUST be less than topfraction !!
\renewcommand{\dblfloatpagefraction}{0.7}	% require fuller float pages

\begin{document}
\maketitle
\pagenumbering{arabic}



\chapter*{Dankwoord}\label{dankwoord}
\addcontentsline{toc}{chapter}{Dankwoord}

De tekst voor het optionele dankwoord.

\chapter*{Voorwoord}\label{voorwoord}
\addcontentsline{toc}{chapter}{Voorwoord}

De tekst voor het optionele voorwoord.

\chapter*{Samenvatting}\label{samenvatting}
\addcontentsline{toc}{chapter}{Samenvatting}

De tekst voor de verplichte samenvatting. Hou het
\href{https://overheid.vlaanderen.be/communicatie/heerlijk-helder}{Heerlijk
Helder}.

\chapter*{Aanbevelingen voor beheer en/of
beleid}\label{aanbevelingen-voor-beheer-enof-beleid}
\addcontentsline{toc}{chapter}{Aanbevelingen voor beheer en/of beleid}

Verplicht wanneer relevant.

\benglish

\chapter*{English abstract}\label{english-abstract}
\addcontentsline{toc}{chapter}{English abstract}

Insert a translation of the summary here. \eenglish

\bfrench

\chapter*{Résumé français}\label{ruxe9sumuxe9-franuxe7ais}
\addcontentsline{toc}{chapter}{Résumé français}

Ajoutez éventuellement une traduction du résumé ici.

\efrench


\clearpage

\phantomsection
\addcontentsline{toc}{chapter}{\contentsname}
\setcounter{tocdepth}{1}
\tableofcontents

\clearpage

\phantomsection
\addcontentsline{toc}{chapter}{\listfigurename}
\listoffigures
\vspace{34pt}

\phantomsection
\addcontentsline{toc}{chapter}{\listtablename}
\listoftables

\clearpage


\chapter{Inleiding}\label{inleiding}

Dit document geeft uitleg bij de update van de kwetsbaarheidskaart
riooloverstorten. De update werd uitgevoerd in 2023-2024 door INBO. De
versie die geüpdate werd, kreeg haar voorgaande update in 2018
(Referentie). Verder in dit document wordt verduidelijkt wat de
definitie is van de waterloopklassen in de versie uit 2018, hoe deze
definitie werd aangepast en welke informatie daarvoor gebruikt werd. Die
informatie werd gehaald uit andere beschikbare kaartlagen.

\emph{De kaart is bedoeld om \ldots{}} \emph{Hier background info over
de reden van een kwetsbaarheidskaart overstorten, met vermelding van:}

\emph{- de richtlijnen die aansturen op een aanpak van de
overstortwerking (EKRW \& ERSA, \ldots{} )} \emph{- de rationale van het
bestaan van een kwetsbaarheidskaart overstorten en dat deze opgesteld
wordt ter bescherming van de ecologie}. \emph{- De beschermde vissoorten
en welke niet in de kwetsbaarheidskaart OS zaten, maar wel in andere
kwetsbaarheidskaarten} \emph{- De kwetsbaarheid is louter en alleen
gebaseerd op de kwetsbaarheid van de oppervlaktewaterlopen in
Vlaanderen. Deze houdt ook geen} \emph{rekening met andere drukken zoals
diffuse verontreiniging, illegale lozingen, versnippering van het
habitat, drukken op de} \emph{(hydro)morfologie, landgebruik,
drinkwatercaptatie en industriële activiteiten.} \emph{- Voor het
opstellen en updaten van de kaart wordt wel rekening gehouden met andere
maatregelen ter bescherming van de ecologie in de oppervlaktewaterlopen,
zoals de speciale beschermingszones (SBZ gebieden).}

\chapter{Huidige versie kwetsbaarheidskaart
riooloverstorten}\label{huidige-versie-kwetsbaarheidskaart-riooloverstorten}

\section{De kaart}\label{de-kaart}

De kwetsbaarheidskaart riooloverstorten uit 2018 gebruikt als basis de
Vlaams Hydrografische Atlas (VHA). De versie die voor de kaarten in dit
document gebruikt werd is \ldots(VERSIE TOEVOEGEN, VRAAG NA BIJ CARINE),
en kan teruggevonden worden op geopunt Vlaanderen (LINK IN TE VOEGEN).

De waterlopen in de VHA zijn opgebouwd uit segmenten. Ieder segment werd
een bepaalde kwetsbaarheid toegekend. Bij de kwestbaarheidskaart
overstorten betreft het de ecologische kwetsbaarheid voor
riooloverstortwerking. De klassen die een waterloopsegment kan hebben
zijn:

\begin{itemize}
\tightlist
\item
  Uiterst kwetsbaar (klasse 1)
\item
  Kwetsbaar (klases 2)
\item
  Strategisch belangrijk voor ecologie (klasse 3)
\item
  Geen van bovenstaande (niet geklasseerd)
\end{itemize}

De definitie van deze klassen wordt in de volgende paragraaf uitgelegd.

Tabel \ref{tab:tabel-KKOS2018} geeft weer hoe alle segmenten van de VHA
verdeeld zijn over de vier bovenstaande klassen.

\begin{longtable}[]{@{}
  >{\raggedright\arraybackslash}p{(\columnwidth - 8\tabcolsep) * \real{0.1812}}
  >{\raggedright\arraybackslash}p{(\columnwidth - 8\tabcolsep) * \real{0.2174}}
  >{\raggedright\arraybackslash}p{(\columnwidth - 8\tabcolsep) * \real{0.3333}}
  >{\raggedright\arraybackslash}p{(\columnwidth - 8\tabcolsep) * \real{0.2464}}
  >{\raggedright\arraybackslash}p{(\columnwidth - 8\tabcolsep) * \real{0.0217}}@{}}
\caption{\label{tab:tabel-KKOS2018}Aantal waterloopsegmenten per klasse in
de kwetsbaarheidskaart riooloverstorten van 2018}\tabularnewline
\toprule\noalign{}
\begin{minipage}[b]{\linewidth}\raggedright
Kwetsbaarheidsklasse
\end{minipage} & \begin{minipage}[b]{\linewidth}\raggedright
Aantal waterloopsegmenten
\end{minipage} & \begin{minipage}[b]{\linewidth}\raggedright
Procentueel aantal waterloopsegmenten (\%)
\end{minipage} & \begin{minipage}[b]{\linewidth}\raggedright
Aantal kilometer waterloop (km)
\end{minipage} & \begin{minipage}[b]{\linewidth}\raggedright
NA
\end{minipage} \\
\midrule\noalign{}
\endfirsthead
\toprule\noalign{}
\begin{minipage}[b]{\linewidth}\raggedright
Kwetsbaarheidsklasse
\end{minipage} & \begin{minipage}[b]{\linewidth}\raggedright
Aantal waterloopsegmenten
\end{minipage} & \begin{minipage}[b]{\linewidth}\raggedright
Procentueel aantal waterloopsegmenten (\%)
\end{minipage} & \begin{minipage}[b]{\linewidth}\raggedright
Aantal kilometer waterloop (km)
\end{minipage} & \begin{minipage}[b]{\linewidth}\raggedright
NA
\end{minipage} \\
\midrule\noalign{}
\endhead
\bottomrule\noalign{}
\endlastfoot
Uiterst kwetsbaar & 1208 & 3 & 846 & 4 \\
Kwetsbaar & 540 & 1 & 491 & 2 \\
Strategisch belangrijke & 3979 & 9 & 2718 & 12 \\
Niet geklasseerd & 40117 & 88 & 19140 & 83 \\
\end{longtable}

Figuur \ref{fig:uiterst-kwetsbaar-2018} geeft de kaart van de uiterst
kwetsbare waterlopen (klasse 1), zoals vastgelegd in de update van 2018.
Dat zijn de waterlopen die binnen Speciale BeschermingsZone (SBZ) liggen
of twee SBZ gebieden verbinden als deze verbinding \textless{} 5 km
bedraagt. De SBZ gebieden zijn opgenomen in de specifieke
instandhoudingsdoelstellingen voor beekprik, rivierdonderpad en habitat
type 3260. Ook doorstroomvijvers met oligotrofe en mesotrofe habitats
worden als `uiterst kwetsbaar' gedefinieerd.

De figuur toont hier de relatie met de habitatrichtlijn en SBZ gebieden.

\begin{figure}

{\centering \includegraphics{pauwels_et_al_kwtsbrhdskrtrlvrstrt_files/figure-latex/uiterst-kwetsbaar-2018-1} 

}

\caption{De kwetsbaarheidskaart riooloverstorten volgens de update in 2018. De gele en groene vlakken zijn de habitatrichtlijn en SBZ gebieden, respectievelijk. De kleuren in de legende (speerpuntklassen) wijzen op de kaartlaag speerpuntgebieden en geven aan tot welke klasse een speerpuntgebied behoort.}\label{fig:uiterst-kwetsbaar-2018}
\end{figure}

\section{De definitie van de
kwetsbaarheidsklassen}\label{de-definitie-van-de-kwetsbaarheidsklassen}

\subsection{Uiterst kwetsbare
waterloopsegmenten}\label{uiterst-kwetsbare-waterloopsegmenten}

Ecologisch uiterst kwetsbare waterloopsegmenten zijn blauw ingekleurd op
de kwetsbaarheidskaart. Het betreft waterloopsegmenten met:
Waterlooptraject(en) gelegen binnen een speciale beschermingszone en
Waterlooptraject(en) die 2 speciale beschermingszones met elkaar
verbindt en waarvan de lengte niet meer bedraagt dan 5 km. De speciale
beschermingszone is opgenomen in de gewestelijke
instandhoudingsdoelstellingen voor beekprik, rivierdonderpad of habitat
3260. Ze zijn onderhevig aan strengere milieudoelstellingen voor
waterkwaliteit in het stroomgebiedbeheerplan, en het betreft ook
doorstroomvijvers met oligo- of mesotrofe waterhabitats. De doelstelling
van de klasse uiterst kwetsbare waterlopen is het bereiken van de
habitatkwaliteit voor de beekprik, rivierdonderpad of habitat 3260
(bron: code van de goede praktijk).

\subsection{Kwetsbare
waterloopsegmenten}\label{kwetsbare-waterloopsegmenten}

Ecologisch kwetsbare waterloopsegmenten zijn groen ingekleurd op de
kwetsbaarheidskaart. Het betreft wterloopsegmenten met beekprik,
rivierdonderpad of habitat 3260 hoofdzakelijk gelegen buiten de speciale
beschermingszone maar wel aangemeld binnen de desbetreffende speciale
beschermingszone. Wanneer het groen ingekleurde (kwetsbare)
waterloopsegment binnen de speciale beschermingzone gelegen is, dan is
noch beekprik, rivierdonderpad of habitat type 3260 er waargenomen.De
doelstelling van de klasse kwetsabre waterlopen is om op termijn de
habitatkwaliteit voor beekprik, rivierdonderpad of habitat 3260 te
bereiken (bron: code van de goede praktijk).

\subsection{Ecologisch strategisch belangrijke
waterloopsegmenten}\label{ecologisch-strategisch-belangrijke-waterloopsegmenten}

Ecologisch strategisch belangrijke waterloopsegmenten zijn geel
ingekleurd op de kwetsbaarheidskaart. Het betreft waterloopsegmenten die
uitmonden in een blauw of een groen traject. Indien een overstort
aanwezig is op een geel traject kan dit stroomafwaarts op een traject
met een hoger beschermingsniveau een schadelijk effect uitoefenen. Deze
trajecten hebben voornamelijk een knipperlichtfunctie (bron: code van de
goede praktijk).

\subsection{Overige
waterloopsegmenten}\label{overige-waterloopsegmenten}

Deze waterloopsegmenten zijn niet ingekleurd op de kwetsbaarheidskaart
omdat ze niet geklasseerd zijn als uiterst kwetsbaar, kwetsbaar of
ecologisch strategisch belangrijk. Dat betekent dat ze niet binnen een
speciale beschermingszone liggen, noch dat beekprik, rivierdonderpad
en/of habitat type 3260 er waargenomen zijn.

\section{Informatie (kaartlagen) aan de basis van de versie van
2018}\label{informatie-kaartlagen-aan-de-basis-van-de-versie-van-2018}

\emph{eventueel in kaarten te tonen of anders deze sectie wegdoen}

\section{Aanpassing van de definities voor de
update}\label{aanpassing-van-de-definities-voor-de-update}

\emph{Uitleg over hoe we de definitie aanpasten en waarom}

\section{Extra kaartlagen}\label{extra-kaartlagen}

In dit digitale rapport kunnen in verschillende figuren extra kaartlagen
aan of uitgezet worden. Dat doe je met de knoppen rechts boven in een
kaart. Deze kaartlagen bevatten informatie over o.a. het speerpuntgebied
waarin een bepaald waterloopsegment ligt, of het (deel)bekken of het
zuiveringsgebied.

In deze kaarten kan je ook kiezen uit verschillende achtergrond
kaartlagen, zoals een orthofoto of een topografische kaart.

De legende bij de kaarten in dit rapport gaat over de speerpuntgebieden
(links onder in de kaarten). De kleuren van de andere kaartlagen staan
in de hoofding van de figuur uitgelegd.

Figuur \ref{fig:base-map-extended-show} toont deze extra kaartlagen.

\begin{figure}

{\centering \includegraphics{pauwels_et_al_kwtsbrhdskrtrlvrstrt_files/figure-latex/base-map-extended-show-1} 

}

\caption{Kaart van Vlaanderen met selecteerbare (aan of uit) kaartlagen voor de rivierbekkens, deelbekkens, habitatrichtlijn gebieden (HRgebieden), Speciale Beschermings Zones (SBZgebieden), de Zuiveringsgebieden van de rioolbeheerders en de Speerpuntgebieden}\label{fig:base-map-extended-show}
\end{figure}

\appendix

\chapter{Eerste hoofdstuk van de
bijlage}\label{eerste-hoofdstuk-van-de-bijlage}

Inhoud van de eerste bijlage

\end{document}
